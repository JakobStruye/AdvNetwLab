The purpose of the Acknowledgement is to let the sender of an IEEE 802.11 Data frame know it was received properly and no retransmission will be needed. For example in the packets mentioned above, the Acknowledgement's receiver address is the Data frame's transmitter address: the receiver of the Data frame lets the sender know it was received. Note that there is no transmitter address in the Acknowledgements. This is related to interframe spaces in IEEE 802.11. There is a well-defined window within which the Acknowledgement for a specific frame must be sent, and it ends before a new data frame can be sent. This means a valid Acknowledgement must always be for the latest Data frame sent: no further identification is necessary.
%The purpose of each of these frames:
%\begin{itemize}
%\item Action, ie 4380 : Action Frames are a type of  management frame used to trigger an action in the cell. In this case, a request to group ACK's (add block ack request) into frames before sending them.
%\item Request-To-Send packet 2294 : RTS frames are used in the RTS/CTS mechanism and allow stations to request the medium for an amount of time. During that time, the station will be able to send uninterrupted and / or without another station becoming active and causing interference.
%\item Clear-To-Send packet 2295 : CTS confirms a RTS request, granting a station the medium for some amount of time.
%\item 802.11 Block Ack 2544 : Acknowledgements of multiple frames, grouped together into one block ack packet.
%\item Acknowledgement, ie 2259 : used to acknowledge the reception of a frame or packet.
%\item Probe Response ie 2601 : Response to a probe request, contains the SSID of the network, information about rates, the kind of network (ad hoc or managed by AP) and the mac address of the AP if there is one, along with various other statistics about the network, in this case about eduroam.
%\item Probe Request ie 2740 : Packet broadcast by a station that wants to know what SSID's are available. It is possible for the wanted SSID to be specified in the probe request, in which case we would call it a directed probe request. In this case, the request was simply broadcast.
%\item data = icmp ??
%\item neighbour discovery stuff?
%\item VHT NDP Announcement ie 716 : 802.11ac specific, used in the VHT sounding procedure and beamforming.
%\end{itemize}
