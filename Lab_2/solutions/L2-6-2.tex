Comparing the packets in 1611 (RTS) and 1612 (CTS), we notice that the RTS contains both a transmitter and a receiver address, while the CTS contains only the latter, making it 6 bytes shorter. When the AP receives a RTS, it has to know which station sent it. When a station receives a CTS after sending a RTS, it can just assume it was sent from the AP the RTS was sent to. The transmitter address is omitted to save 6 bytes. Less importantly, the type/subtype field is obviously different. Unrelated to this being a RTS and a CTS, the FCS and duration fields differ.
