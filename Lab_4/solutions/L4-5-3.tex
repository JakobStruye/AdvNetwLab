There are no significant differences between the authentication phases (packets 409 and 411 for WEP, 176 and 178 for WPA2). For the assocation phase (packets 424 and and 436 for WEP, 189 and 191 for WPA2), an additional 22 byte RSN (Robust Security Network) Information tag is present in WPA2's Request. This identifies the requested protocols. Other than this there are no significant differences. \\ \\The main difference is that this is followed with 4 EAPOL (Extensible Authentication Protocol over LAN) packets which perform the actual WPA2 authentication. This is known as EAPOL's 4-way handshake. Station and AP exchange some randomly generated values which are used in combination with the passphrase to generate the Pairwise Transient Key (PTK) they will use for encryption. Someone overhearing this cannot recreate the PTK unless they also know the passphrase. A step like this is completely absent in WEP (with Open System authentication): the AP assumes the station knows the passphrase, knowing the station will not be able to encrypt/decrypt communication without it.
